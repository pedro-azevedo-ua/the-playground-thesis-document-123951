\chapter{Conclusion}

The scenario of the Portuguese basketball community reveals a genuine coordination gap in the amateur sport, where players remain disconnected despite sharing the same passion. Digital transformation in team sports is yet to be accomplish, and this dissertation proposes a bridge that gap, through the design and development of a Kotlin Multiplatform mobile application, supported with Supabase in the backend, that enables amateur basketball players to organise casual games, check court availability and build a sustainable community of practice around public courts in Portugal.

By combining User-Centred Design principles, to shape the platform for the Portuguese user, with a carefully selected technology stack and Agile development methodology, this project is positioned to deliver both functional and production-ready application. With user analysis, requirements established, architectural design, and implementation planning now in place, the implementation and evaluation phases will demonstrate whether a thoughtfully designed digital platform can truly transform how casual athletes coordinate and sustain engagement with their sport, connecting Lisbon's fragmented basketball communities while establishing a foundation for broader adoption across Portugal.