\chapter{Phase 1 - Prototype and Feature Validation}

In the first phase of the project, the main goal it to design the application ending up with a prototype and the users evaluation, to validate the features and their importance. In this chapter it will be describe the implementation of this research and prototyping phase, starting by understanding and specifying the context of use, and then gathering the requirements. 

\section{Understand and specify the context of use }

The first part that is important to design the platform is to understand who is the target user and how will the application be used. For this, a user research was conducted through a survey targeting basketball players in Portugal, then two persona where made based on the survey and then user scenarios.

\subsection{User research}\label{subsubsec:user-research}

A user research was conducted through a survey in order to understand who goes to public basketball courts, how often they go, why they are not going more often, what gets them annoyed and what would they wish to improve in their experience playing in basketball courts. To collect data, it was used Mircrosoft Forms (see questions in Appendix \ref{appendix:survey}) and  was shared through \textit{Whatsapp}, between different people and groups and a flyer was made and distributed in a public basketball court in Lisbon (see flyer in Appendix \ref{appendix:flyer}). A total of 50 responses were gather and here are some takeaways:

\begin{itemize}
	\item 80\% of the responses were from men;
	\item  27 responses (54\%)  are from people between 18-25 years old, and 16 (32\%) from 26-35 years old;
	\item The majority live in Lisbon, with 10 (20\%) responses from Setúbal, 2 (4\%) from Aveiro and 1 (2\%) from Porto;
	\item 27 (46\%)people that responded work full time and 21 (36\%) are studying in university;
	\item When asked if they play basketball, 34 (40\%) said that they play in public courts causal games, 21 (25\%) go to private courts, 15 (18\%) just shoot some balls and 6 (7\%) play in a competitive team;
	\item 39 (78\%) started playing before they were 16 years old;
	\item 15 participants go to play in Alcântara (24\%) public court and 14 (22\%) in Moinho do Penedo in Monsato public court, both in Lisbon;
	\item When asked the Frequency that they go to public courts, 18 (35\%) said that go once a week and 14 (27\%) responded with seasonal, with 88\% wanting to go more often;
	\item 24 (24\%) do not go more often because of time constraints and 22 (22\%) said that they do not known enough people to play with, besides that, 10 (10\%) said that nobody is in the court when they go;
	\item Usually, 30 (27\%) participants go with friends to play against each other, 33 (29\%) go with friends to play against others, 19 (17\%) go alone to shoot some balls, but are open to play some games if it comes up, and 14 (13\%) just want to go shoot some balls;
	\item 58\% want more competition, while the other 42\% are satisfied;
	\item When asked to the participants what they find more annoying in the public courts, most complain about the conditions of the court, and other about the players, that some are unpleasant to play with, the level of competitiveness is not match, and lack of organization;
\end{itemize}

\subsection{Persona}

A Persona is a fictional character created based on the user research that represents the different user types. Personas help to understand the user needs, guiding the next phases of the development process. For this project, 2 types of users were found, creating 2 different Personas.

\subsubsection{Persona 1: Miguel — The Competitive Player}

\textbf{Persona} - There are basketball players that want to play against other players, have more competition, as seen in the survey with 58\%, and as 21 participants do not play more often because they do not know enough people, and 10 complain that nobody is there when they go, it can be assume that they want to find other players to play with. Out of this analysis, the first persona emerges, Miguel the competitive  player, that wants to play competitive games more often, the Fig\ref{fig:persona-miguel} represents that persona.

\begin{figure}[h]
	\centering
	\includegraphics[width=0.9\textwidth]{figs/persona-miguel}
	\caption{Persona 1 Miguel}
	\label{fig:persona-miguel}
\end{figure}

Miguel usually goes with friends to play against other players that they find on the court, however they never know if a good amount competitors will be there, at the same time as them, maybe there is someone, but can be there for a long time already and will leave soon, or does not want to play with others, and there is also the possibility the players encounter are not as good as Miguel and is not that fun to play with. Because of this, Miguel does not go more often, as it can be a waste of time. So, he needs a platform to meet other players that want to play a competitive game, or to just see how many players are in the court to play a spontaneous game. Miguel also wants to know if the players are the right match for him, if they play in the same level and if they will dedicate the same as him. 

Below are two different scenarios that can happen to players like Miguel.


\textbf{Scenario 1: Miguel plays on Sunday} Miguel usually goes on Sunday mornings to a court in Alcântara with two friends to play pickup games against other players who might be there. To play a 3v3 game, they need at least three more players. For a 4v4, they need five more, joining someone to complete their team. Typically, Sunday mornings bring older players to the court, and Miguel and his friends often end up joining their games. However, this is not the type of competition they want, so sometimes they try to go at a different time to find players with similar skill levels and physical condition. Occasionally, they encounter another group of friends around their age and level and challenge them to a game. Sometimes they find competitive teams, but other times they end up with players who are just fooling around or with team-mates who do not pass the ball, leaving Miguel and his friends frustrated.

\textbf{Scenario 2: Miguel plays after work on Summer} - During Summer, Miguel likes to play after work. But when he and his friends arrive at the court, no one is there, so they just shoot some hoops and don’t have as much fun or get as much running as they hoped. When they are about to leave, a pair of friends sometimes arrives, also hoping to find other players to challenge. However, by then Miguel and his friends are almost ready to go. If they had organized in advance or somehow known that others were coming, they might have adjusted their schedule and been able to play together.

\textbf{How the platform would help} - With a platform, Miguel would proceed as follows: He opens the application and creates a game, a 4v4, set for Saturday at 3 p.m., and invites his two friends, who accept. Miguel now needs five more players, but if he can’t find a fifth, he can reduce the game to a 3v3. He makes the game public, so anyone interested in playing a 3v3 or 4v4 receives a notification, or they see it in the list of upcoming games in the application. The first rival joins after seeing Miguel team’s ranking. Soon, a team of four expresses intent to join, and so, Miguel invites the first adversary to his team and accepts the other team as opponents.
On Saturday afternoon, the group arrives at the available court and checks in via the application. Other people now know there are eight players present, and the game appears in the application as happening live at that location. Miguel’s team wins, and a member from each team records the result in the application. Because the game is ranked, every member of the winning team earns points toward their individual rankings. Miguel enjoyed playing with their fourth team-mate, so they decide to register the team and enter the league table for teams of four. The next time Miguel wants to play, he can create a game with his new team.

\subsubsection{Persona 2: Carlos — The Flexible Shooter}

\textbf{Persona} -The other type of user, is the player that just goes to the court to practice and shoot some balls, as 29 players out of 50 usually go alone for that, some do not mind playing a game, this type of user wants to find a court or time slot to relax, without competition and space practice. In the Fig \ref{fig:persona-carlos} is the second persona, Carlos the flexible shooter, he goes once a week to a public court to shoot around, and if someone asks him to play, he accepts the invitation. Some times the court is full and Carlos does not have space to shoot and everyone is playing too aggressive. He needs a platform to see which courts have space to just shoot practice. 


\begin{figure}[h]

	\centering
	\includegraphics[width=0.9\textwidth]{figs/persona-carlos}
	\caption{Persona 1 Carlos}
	\label{fig:persona-carlos}
	
\end{figure}

\textbf{Scenario: Carlos go practice Saturday morning} - These days, Carlos enjoys going to the court to practice his shot. He goes at least once a week, and when other players invite him to join a pickup game, he’s always ready for the challenge. Carlos prefers to play on Saturday mornings at Monte do Penedo in Monsanto. However, especially when the weather is good, the court can become crowded, with several games going on at once and multiple people shooting at the same basket. During these times, Carlos often wishes he had gone to a different court.

\textbf{How the platform would help} - With a platform, Carlos could check court availability before leaving home, allowing him to choose a less crowded court. He could also discover courts he did not know about and read reviews from other players about court conditions. Before heading out, Carlos could use the application to indicate his intention to go to a specific court at a certain time, specifying that he plans to shoot around but is open to joining non-competitive games. Alternatively, he could check in when he arrives, letting others know he’s available for practice or casual play.

\section{Specifying the user requirements} 

\subsection{Epics and User Stories}
From the personas and the user scenarios requirements where gather.
The User Stories were organised into five groups or, as called in Scrum, Epics:
\begin{itemize}
	\item \textbf{PLY-05 Game Management} (Table \ref{tab:epic-game-management}) encompasses all features related to creating, discovering, joining, and managing basketball games.
	\item \textbf{PLY-06 Court Availability} (Table \ref{tab:epic-court-availability}) handles court discovery, management, occupancy tracking, and user check-ins.
	\item \textbf{PLY-07 Players and Social Features} (Table \ref{tab:epic-players-social}) encompasses player discovery, friendship management, communication, and player interactions.
	\item \textbf{PLY-08 Ranking and Team System} (Table \ref{tab:epic-ranking-team}) manages competitive ranking, team creation, and team-based competitions.
	\item \textbf{PLY-09 User Profile and Settings} (Table \ref{tab:epic-profile-settings}) handles authentication, profile management, notification preferences, and user data.
\end{itemize}

\begin{table}[h]
	\centering
	\caption{Epic: Game Management - User Stories}
	\label{tab:epic-game-management}
	\begin{tabular}{p{2cm}p{4cm}p{8cm}}
		\toprule
		\multicolumn{3}{c}{\textbf{PLY-05 Game Management}} \\
		\midrule
		\textbf{ID} & \textbf{Title} & \textbf{User Story} \\
		\midrule
		PLY-10 & Create a Game & ``As Miguel, I want to create a game with custom parameters, so that I can organize matches with specific settings and invite players.'' \\
		\midrule
		PLY-41 & View Game Details & ``As Miguel, I want to view complete details of a game I'm in or interested in, so that I know when to show up, who's playing, and what to expect.'' \\
		\midrule
		PLY-11 & Receive Game Notifications & ``As Miguel, I want to receive notifications when games are created in my favourite courts, so that I can quickly join games that interest me.'' \\
		\midrule
		PLY-12 & Respond to Game Invitation & ``As a basketball player, I want to accept or decline game invitations, so that I can control which games I participate in.'' \\
		\midrule
		PLY-13 & Browse and Search Games & ``As Miguel, I want to browse and filter available games, so that I can find games that match my preferences.'' \\
		\midrule
		PLY-14 & Request to Join Game & ``As Miguel, I want to request to join public games, so that I can participate in games organized by others.'' \\
		\midrule
		PLY-15 & Manage Join Requests and Players & ``As Miguel and game host, I want to accept or reject players requesting to join my game and manage players in game list, so that I can control who participates.'' \\
		\midrule
		PLY-17 & Start Game & ``As Miguel and game host, I want to start the game when all players arrive, so that we can begin playing and tracking the match.'' \\
		\midrule
		PLY-18 & End Game & ``As Miguel, I want to end the match and record the final score and winning team, so that results are tracked for rankings and history.'' \\
		\midrule
		PLY-16 & Challenge Teams & ``As Miguel and a team member, I want to challenge other teams to a game, so that we can compete against organized opponents.'' \\
		\midrule
		PLY-19 & Spectator Mode & ``As Carlos, I want to mark myself as watching a game and record the result, so that I can contribute to game accuracy even if I'm not playing.'' \\
		\midrule
		PLY-25 & Spontaneous Game Creation & ``As Carlos, while I am in the court, I want to quickly create an instant game, so that me and other player in the court can play without detailed planning, but still counting in the app.'' \\
		\midrule
		PLY-24 & Tournament Creation & ``As Miguel, I want to create tournaments with multiple teams and rounds, so that I can run structured competitive events and include different players and teams.'' \\
		\bottomrule
	\end{tabular}
\end{table}

\begin{table}[h]
	\centering
	\caption{Epic: Court Availability - User Stories}
	\label{tab:epic-court-availability}
	\begin{tabular}{p{2cm}p{4cm}p{8cm}}
		\toprule
		\multicolumn{3}{c}{\textbf{PLY-06 The Court Availability}}\\
		\midrule
		\textbf{ID} & \textbf{Title} & \textbf{User Story} \\
		\midrule
		PLY-40 & View Court Details and Add to Favourites & ``As Miguel, I want to view court details and add courts to my favorites, so that I can easily find courts I like and receive notifications about games there.'' \\
		\midrule
		PLY-42 & Search Courts & ``As Carlos, I want to search for courts interactively and easily, so that I can find new courts to try out.'' \\
		\midrule
		PLY-26 & Court Check-In & ``As Carlos, I want to check in when I arrive at a court, so that others know I'm there.'' \\
		\midrule
		PLY-27 & Court Occupancy Display & ``As Carlos, I want to see how many players are at each court, so that I can decide where to play.'' \\
		\midrule
		PLY-28 & Add New Courts & ``As Carlos, I want to add courts that aren't in the system, so that the app includes all court locations.'' \\
		\midrule
		PLY-29 & Rate and Review Courts & ``As Carlos, I want to rate and review courts, so that others have accurate and updated information about court quality.'' \\
		\midrule
		PLY-30 & Announce Playing Intentions & ``As Miguel, I want to announce when I plan to play at a court, so that others can join me.'' \\
		\midrule
		PLY-31 & View Expected Court Availability & ``As Carlos, I want to see how many players plan to be at a court at specific times, so that I can plan my sessions better.'' \\
		\bottomrule
	\end{tabular}
\end{table}

\begin{table}[h]
	\centering
	\caption{Epic: Players and Social Features - User Stories}
	\label{tab:epic-players-social}
	\begin{tabular}{p{2cm}p{4cm}p{8cm}}
		\toprule
		\multicolumn{3}{c}{\textbf{PLY-07 The Players and Social Features}}\\
		\midrule
		\textbf{ID} & \textbf{Title} & \textbf{User Story} \\
		\midrule
		PLY-21 & View Player Profiles & ``As Miguel, I want to view other players' profiles, so that I can find potential teammates or opponents.'' \\
		\midrule
		PLY-32 & Send friend Requests & ``As Miguel, I want to add other players as friends, so that I can easily organize games with them and know when they are at the court.'' \\
		\midrule
		PLY-33 & Respond to friend Requests & ``As Miguel, I want to accept or decline friend requests, so that I can manage my friends list.'' \\
		\midrule
		PLY-36 & Search Players and Teams & ``As Miguel, I want to search for other players and teams, so that I can connect with them or issue challenges.'' \\
		\midrule
		PLY-34 & Game Chat & ``As Miguel, I want to chat with other players in my game, so that we can coordinate details and communicate.'' \\
		\bottomrule
	\end{tabular}
\end{table}

\begin{table}[h]
	\centering
	\caption{Epic: Ranking and Team System - User Stories}
	\label{tab:epic-ranking-team}
	\begin{tabular}{p{2cm}p{4cm}p{8cm}}
		\toprule
		\multicolumn{3}{c}{\textbf{PLY-08 The Ranking and Team System}}\\
		\midrule
		\textbf{ID} & \textbf{Title} & \textbf{User Story} \\
		\midrule
		PLY-20 & Player Ranking System & ``As Miguel, a competitive player, I want to earn points for victories and see my ranking, so that I can track my progress and compete with others.'' \\
		\midrule
		PLY-22 & Team Management & ``As Miguel, I want to create and manage teams, so that I can play regularly with the same group and compete in team rankings.'' \\
		\bottomrule
	\end{tabular}
\end{table}

\begin{table}[h]
	\centering
	\caption{Epic: User Profile and Settings - User Stories}
	\label{tab:epic-profile-settings}
	\begin{tabular}{p{2cm}p{4cm}p{8cm}}
		\toprule
		\multicolumn{3}{c}{\textbf{PLY-09 The User Profile and Settings}}\\
		\midrule
		\textbf{ID} & \textbf{Title} & \textbf{User Story} \\
		\midrule
		PLY-37 & User Authentication & ``As Carlos, I want to register, log in, and recover my password, so that I can securely access my account.'' \\
		\midrule
		PLY-38 & Profile Management & ``As Carlos, I want to edit my profile information, so that my details stay updated.'' \\
		\midrule
		PLY-35 & View Game History & ``As Miguel, I want to view my past games, so that I can track my playing activity and review results.'' \\
		\midrule
		PLY-39 & Notification Preferences & ``As Carlos, I want to customize which notifications I receive, so that I only get alerts that matter to me.'' \\
		\bottomrule
	\end{tabular}
\end{table}

\subsection{Non-Functional Requirements}

For the non-functional requirements, they were aligned with ISO/IEC 25010\cite{ISO25010} quality model. 
Each requirement specifies quality attributes and constraints that apply across multiple functional features.
In the ISO there are nine quality characteristics\cite{ISO25010}:
\begin{itemize}
	\item Functional Suitability
	\item Performance Efficiency
	\item Compatibility
	\item Interaction Capability
	\item Reliability
	\item Security
	\item Maintainability
	\item Flexibility
	\item Safety
\end{itemize}

Below is a list of the non-functional requirements created to set guidelines to follow the International Standard guaranteeing software quality.  


% Reliability NFRs
\begin{table}[h]
	\centering
	\caption{Non-Functional Requirements: Reliability}
	\label{tab:nfr-reliability}
	\begin{tabular}{p{2cm}p{4cm}p{7cm}p{2cm}}
		\toprule
		\multicolumn{4}{c}{\textbf{Reliability}}\\
		\midrule
		\textbf{ID} & \textbf{Title} & \textbf{Description} & \textbf{Priority} \\
		\midrule
		NFR-01 & Offline Availability & The application shall support offline access to previously loaded data, informing when the user is offline and the last updated timestamp. & Must Have \\
		\midrule
		NFR-02 & Data Consistency & The application shall not lose data after a write operation. Any data acknowledged to the user as saved shall persist across application crashes, OS termination, or device restart & Must Have \\
		\midrule
		NFR-03 & Graceful Degradation & In case of a failed backend request, the application shall: (1) Display cached data when available, (2) Show a clear, user-friendly error message, (3) Never crash or display technical error details, (4) Allow retry when appropriate. & Should Have \\
		\midrule
		NFR-23 & Session State Persistence & The application shall preserve the user session state across application restarts using secure authentication tokens. Sessions shall persist until the user explicitly logs out or until tokens expire after a prolonged period of inactivity. & Must Have \\
		\midrule
		NFR-24 & Data Sync on App Resume & The application shall automatically synchronize data with the backend when resumed after being backgrounded. Data is considered fresh for 5 minutes, if more than 5 minutes have passed, a sync is triggered. & Should Have \\
		\midrule
		NFR-26 & Background Sync & The application may implement periodic background synchronization to keep user data up-to-date even when the app is not actively in use. & Could Have \\
		\midrule
		NFR-27 & Form State Recovery & The application shall recover form state when the app is backgrounded during form entry. & Should Have \\
		\bottomrule
	\end{tabular}
\end{table}

\vspace{1cm}

% Security NFRs
\begin{table}[h]
	\centering
	\caption{Non-Functional Requirements: Security}
	\label{tab:nfr-security}
	\begin{tabular}{p{2cm}p{4cm}p{7cm}p{2cm}}
		\toprule
		\multicolumn{4}{c}{\textbf{Security}}\\
		\midrule
		\textbf{ID} & \textbf{Title} & \textbf{Description} & \textbf{Priority} \\
		\midrule
		NFR-04 & User Authentication & The application shall authenticate users using either social login or email/password. Authentication credentials and session tokens shall be securely stored using platform-provided secure storage mechanisms. & Must Have \\
		\midrule
		NFR-05 & Data Privacy (GDPR Compliance) & The application shall comply with GDPR requirements: explicit user consent before data collection, data minimization, user right to access/delete data, transparent privacy policy. Location data used only for real-time search, not stored long term. & Must Have \\
		\midrule
		NFR-06 & Secure Data Transmission & All data exchanged between the client and backend shall be transmitted over encrypted channels using HTTPS with TLS 1.3. Invalid or self-signed certificates shall be rejected. & Must Have \\
		\midrule
		NFR-07 & Location Privacy & The application shall never expose to other users the user location coordinates; only the location of the court should be shared publicly. & Must Have \\
		\midrule
		NFR-21 & Message Encryption & The application shall implement end-to-end encryption for user messages. Message content shall be encrypted on the sender's device and decrypted only on the recipient's device using industry-standard cryptographic algorithms (e.g., AES-256). & Must Have \\
		\bottomrule
	\end{tabular}
\end{table}

\vspace{1cm}

% Interaction Capability NFRs
\begin{table}[h]
	\centering
	\caption{Non-Functional Requirements: Interaction Capability}
	\label{tab:nfr-interaction}
	\begin{tabular}{p{2cm}p{4cm}p{7cm}p{2cm}}
		\toprule
		\multicolumn{4}{c}{\textbf{Interaction Capability}}\\
		\midrule
		\textbf{ID} & \textbf{Title} & \textbf{Description} & \textbf{Priority} \\
		\midrule
		NFR-08 & Learnability & First-time users shall be able to complete core tasks (finding a court, joining a game, sending a message) without prior training or documentation, supported by intuitive UI design and optional onboarding hints. & Must Have \\
		\midrule
		NFR-09 & Error Feedback & The application shall provide a clear, descriptive error message when a user action fails, explaining why. & Should Have \\
		\midrule
		NFR-10 & Outdoor Readability & The application shall remain readable in outdoor sunlight conditions through high contrast UI and support for system dark/light modes. & Could Have \\
		\midrule
		NFR-25 & UI State Preservation & The application shall preserve user interface state (filters, search terms, scroll positions, form inputs) when the app is backgrounded and the process is not terminated. This state shall be restored when the app is reopened during the same session. & Should Have \\
		\bottomrule
	\end{tabular}
\end{table}

\vspace{1cm}

% Compatibility NFRs
\begin{table}[h]
	\centering
	\caption{Non-Functional Requirements: Compatibility}
	\label{tab:nfr-compatibility}
	\begin{tabular}{p{2cm}p{4cm}p{7cm}p{2cm}}
		\toprule
		\multicolumn{4}{c}{\textbf{Compatibility}}\\
		\midrule
		\textbf{ID} & \textbf{Title} & \textbf{Description} & \textbf{Priority} \\
		\midrule
		NFR-13 & GPS Integration & The application shall integrate with device's native GPS/GNSS hardware for location services. & Must Have \\
		\midrule
		NFR-14 & Maps Service Integration & Application shall integrate with device maps services to open maps court location. & Could Have \\
		\midrule
		NFR-20 & Camera/Storage Integration & The system shall access native camera to take photographs and allow image upload directly from the device camera and storage. & Must Have \\
		\midrule
		NFR-22 & Calendar Service Integration & Application shall integrate with device calendar services to export game events. & Could Have \\
		\bottomrule
	\end{tabular}
\end{table}

\vspace{1cm}

% Flexibility NFRs
\begin{table}[h]
	\centering
	\caption{Non-Functional Requirements: Flexibility}
	\label{tab:nfr-flexibility}
	\begin{tabular}{p{2cm}p{4cm}p{7cm}p{2cm}}
		\toprule
		\multicolumn{4}{c}{\textbf{Flexibility}}\\
		\midrule
		\textbf{ID} & \textbf{Title} & \textbf{Description} & \textbf{Priority} \\
		\midrule
		NFR-11 & Cross-Platform Support & The application shall run on both Android and iOS platforms, supporting Android 8.0 (API 26) and later, and iOS 13 and later. & Must Have \\
		\midrule
		NFR-12 & Screen Compatibility & The application shall function correctly on devices with varying screen sizes and capabilities, for phones only. & Should Have \\
		\bottomrule
	\end{tabular}
\end{table}

\vspace{1cm}

% Maintainability NFRs
\begin{table}[h]
	\centering
	\caption{Non-Functional Requirements: Maintainability}
	\label{tab:nfr-maintainability}
	\begin{tabular}{p{2cm}p{4cm}p{7cm}p{2cm}}
		\toprule
		\multicolumn{4}{c}{\textbf{Maintainability}}\\
		\midrule
		\textbf{ID} & \textbf{Title} & \textbf{Description} & \textbf{Priority} \\
		\midrule
		NFR-15 & Code Modularity & The application shall follow clean architecture principles with separation between presentation, domain, and data layers. Changes to one module shall have minimal impact on others. & Should Have \\
		\midrule
		NFR-16 & Testability & The application shall have unit tests for core business logic. Acceptance tests shall be written for user stories following ATDD methodology. & Should Have \\
		\midrule
		NFR-17 & Documentation & The application shall be complemented with documentation for APIs and business logic. & Should Have \\
		\midrule
		NFR-18 & Logging & The application shall generate centralized logs for failures and errors, structured for analysis. & Should Have \\
		\bottomrule
	\end{tabular}
\end{table}

\vspace{1cm}

% Safety NFRs
\begin{table}[h]
	\centering
	\caption{Non-Functional Requirements: Safety}
	\label{tab:nfr-safety}
	\begin{tabular}{p{2cm}p{4cm}p{7cm}p{2cm}}
		\toprule
		\multicolumn{4}{c}{\textbf{Safety}}\\
		\midrule
		\textbf{ID} & \textbf{Title} & \textbf{Description} & \textbf{Priority} \\
		\midrule
		NFR-19 & User Moderation & The application shall include report features to report players and events to mitigate harassment or unsafe gatherings. & Could Have \\
		\bottomrule
	\end{tabular}
\end{table}

\vspace{1cm}

% Performance Efficiency NFRs
\begin{table}[h]
	\centering
	\caption{Non-Functional Requirements: Performance Efficiency}
	\label{tab:nfr-performance}
	\begin{tabular}{p{2cm}p{4cm}p{7cm}p{2cm}}
		\toprule
		\multicolumn{4}{c}{\textbf{Performance Efficiency}}\\
		\midrule
		\textbf{ID} & \textbf{Title} & \textbf{Description} & \textbf{Priority} \\
		\midrule
		NFR-28 & UI Responsiveness & The application shall respond to user interactions (navigation, button taps, scrolling) within 500 ms under normal operating conditions. & Must Have \\
		\midrule
		NFR-29 & Data Load Time & Application data shall load within 2 seconds when network connectivity is available and within 1 second when using cached data. & Should Have \\
		\midrule
		NFR-30 & Battery Efficiency & The application shall minimize battery consumption by limiting background processing, using location services only when necessary, and adapting update frequency based on application state. & Must Have \\
		\midrule
		NFR-31 & Scalability & The system shall support at least 1,000 registered users and 100 concurrent active users within the Lisbon area without degradation of core functionalities. & Should Have \\
		\bottomrule
	\end{tabular}
\end{table}



